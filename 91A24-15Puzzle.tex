\documentclass[12pt]{article}
\usepackage{pmmeta}
\pmcanonicalname{15Puzzle}
\pmcreated{2013-03-22 16:46:21}
\pmmodified{2013-03-22 16:46:21}
\pmowner{PrimeFan}{13766}
\pmmodifier{PrimeFan}{13766}
\pmtitle{15 Puzzle}
\pmrecord{5}{39001}
\pmprivacy{1}
\pmauthor{PrimeFan}{13766}
\pmtype{Definition}
\pmcomment{trigger rebuild}
\pmclassification{msc}{91A24}
\pmclassification{msc}{00A08}
\pmsynonym{Fifteen Puzzle}{15Puzzle}
\pmsynonym{Game of Fifteen}{15Puzzle}

\endmetadata

% this is the default PlanetMath preamble.  as your knowledge
% of TeX increases, you will probably want to edit this, but
% it should be fine as is for beginners.

% almost certainly you want these
\usepackage{amssymb}
\usepackage{amsmath}
\usepackage{amsfonts}

% used for TeXing text within eps files
%\usepackage{psfrag}
% need this for including graphics (\includegraphics)
%\usepackage{graphicx}
% for neatly defining theorems and propositions
%\usepackage{amsthm}
% making logically defined graphics
%%%\usepackage{xypic}

% there are many more packages, add them here as you need them

% define commands here

\begin{document}
The {\em 15 Puzzle} is a square tablet containing 15 smaller square tiles labeled with the integers 1 to 15, set so that only one square may be moved at a time into the only available empty square by a move up or down or left or right (but never diagonally). The goal of the puzzle is to take a puzzle in an unsorted initial state, such as

\begin{tabular}{|c|c|c|c|}
7 & 8 & & 1 \\
2 & 3 & 4 & 5 \\
6 & 9 & 10 & 11 \\
12 & 13 & 14 & 15 \\
\end{tabular}

and set each tile in its proper order.

\begin{tabular}{|c|c|c|c|}
1 & 2 & 3 & 4 \\
5 & 6 & 7 & 8 \\
9 & 10 & 11 & 12 \\
13 & 14 & 15 & \\
\end{tabular}

The puzzle was invented by Noyes Chapman, who also created a famously unsolvable version with 14 and 15 switched. His original idea was to construct a puzzle with 16 tiles that would be moved to form a magic square with 34 as its magic constant. The 15 Puzzle was initially made of wood; today they are almost always made of plastic. Darling calls it ``the Rubik's cube of its day.''

\begin{thebibliography}{1}
\bibitem{dd} D. Darling, ``15 Puzzle'' in {\it The Universal Book of Mathematics: From Abracadabra To Zeno's paradoxes}. Hoboken, New Jersey: Wiley (2004)
\end{thebibliography}
%%%%%
%%%%%
\end{document}
