\documentclass[12pt]{article}
\usepackage{pmmeta}
\pmcanonicalname{ZerosumGame}
\pmcreated{2013-03-22 16:32:52}
\pmmodified{2013-03-22 16:32:52}
\pmowner{PrimeFan}{13766}
\pmmodifier{PrimeFan}{13766}
\pmtitle{zero-sum game}
\pmrecord{7}{38730}
\pmprivacy{1}
\pmauthor{PrimeFan}{13766}
\pmtype{Definition}
\pmcomment{trigger rebuild}
\pmclassification{msc}{91A99}
\pmsynonym{zero sum game}{ZerosumGame}

% this is the default PlanetMath preamble.  as your knowledge
% of TeX increases, you will probably want to edit this, but
% it should be fine as is for beginners.

% almost certainly you want these
\usepackage{amssymb}
\usepackage{amsmath}
\usepackage{amsfonts}

% used for TeXing text within eps files
%\usepackage{psfrag}
% need this for including graphics (\includegraphics)
%\usepackage{graphicx}
% for neatly defining theorems and propositions
%\usepackage{amsthm}
% making logically defined graphics
%%%\usepackage{xypic}

% there are many more packages, add them here as you need them

% define commands here

\begin{document}
A \emph{zero-sum game} is a game in which only one player can win (that is, achieve the goal of the game) and the losses (the failure to obtain a goal of the game) of any player are matched by gains by another player. A zero-sum game is a finite game (a game that eventually comes to an end), and though only one player can win, the game can also end in a draw (meaning that neither side can win).

For example, in chess, when a player loses a piece captured by another player, the other player gains more open avenues on which to attack the king of the opponent. In Reversi (or Othello), a player must capture at least one of the opponent's pieces; if not, then the player must pass. In poker, all players must contribute to the pot; whoever has the best hand claims the entire pot and the losers lose everything they put in the pot.

Zero-sum games were extensively studied by John von Neumann.
%%%%%
%%%%%
\end{document}
