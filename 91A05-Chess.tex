\documentclass[12pt]{article}
\usepackage{pmmeta}
\pmcanonicalname{Chess}
\pmcreated{2013-03-22 16:39:45}
\pmmodified{2013-03-22 16:39:45}
\pmowner{PrimeFan}{13766}
\pmmodifier{PrimeFan}{13766}
\pmtitle{chess}
\pmrecord{5}{38868}
\pmprivacy{1}
\pmauthor{PrimeFan}{13766}
\pmtype{Topic}
\pmcomment{trigger rebuild}
\pmclassification{msc}{91A05}

% this is the default PlanetMath preamble.  as your knowledge
% of TeX increases, you will probably want to edit this, but
% it should be fine as is for beginners.

% almost certainly you want these
\usepackage{amssymb}
\usepackage{amsmath}
\usepackage{amsfonts}

% used for TeXing text within eps files
%\usepackage{psfrag}
% need this for including graphics (\includegraphics)
%\usepackage{graphicx}
% for neatly defining theorems and propositions
%\usepackage{amsthm}
% making logically defined graphics
%%%\usepackage{xypic}

% there are many more packages, add them here as you need them

% define commands here

\begin{document}
{\em Chess} is a two-player game of skill and strategy played on an 8 by 8 board with a starting complement of 32 pieces, 16 for each side. The players take turns moving one piece at a time according to the established rules for how the pieces may move. The goal of the game is to put the opponent's king in a position where capture is inevitable regardless of whatever move the opponent could make.

Chess is of course of interest to game theorists, but it has also interested mathematicians in other fields and been used to illustrate various problems.

There is the legend of the emperor who, wanting to reward a loyal subject, asked him to name his prize. The subject asked for a chess board with a grain of rice on the first square, twice as many on the second square, twice as many on the third square as on the second, and so on to the sixty-fourth square. At first the king thought it an excessively modest request, but was enraged when the treasurer told him that the kingdom actually lacked the capacity to pay the named prize.

\subsection{Overview of the rules}

There is only one valid initial state for the chessboard and White must move first.

At each turn, a player may move one of his pieces according to what piece that is (the special move of castling being an exception to this rule). If the move concludes on a piece occupied by the opponent's piece, the player captures that piece by removing it from the board. No piece may jump over any other pieces (except the knights, which can jump over any piece of either side but only capture an opponent's piece that is where they land).

When a pawn reaches the last line of defense of the opponent, that pawn can be converted to a queen, bishop, knight or rook, though players usually choose a queen. Thus it's possible for one player to have as many as eight queens.

It is forbidden for a player to make a move that puts his own king in jeopardy. However, it could happen in amateur play that such a move is made and the opponent fails to realize the opportunity, and the game perhaps proceeds to a state that would be impossible under the rules. The goal of the game is to create a situation of inevitable capture for the opponent's king, called a ``checkmate;'' in tournament play the actual capture is not actually performed.

If a player puts the opponent's king in a position where the king will be captured regardless of whether the king moves or stays put and no other pieces can rescue the king, then the attacking player has won. However, if a player puts the opponent's king in a position where the king has to stay put and no other pieces can come to the aid of the king, a stalemate has occurred, neither player won, but spectators will likely see this as a lack of skill on the part of the player who caused the stalemate. If a player captures all the pieces of the opponent except the king, then he must checkmate the king before 13 moves or the game is a draw.

In timed play, additional rules apply, but these are usually not of interest to mathematicians more interested in the logic of the game.

\subsection{Game theory and heuristics}

Since there is only one valid initial state, the game begins with White moving either a pawn or a knight, for a total of only twenty possible moves at the beginning. Likewise Black is limited for a first move. The number of possible moves balloons exponentially afterwards and game theorists are not completely sure how many possible chess games there are under the rules of the game. The game-tree is not completely known.

\subsection{Chess problems}

Mathematicians have studies many problems where certain pieces are placed on an otherwise empty board but are only allowed to move by the rules of the game. Most of these are not situations likely to arise from normal play.

Perhaps the most famous problem of mathematical logic applied to chess is the eight queens problem, in which, in the absence of any other pieces, eight queens are placed on the board such that none of the queens can capture any of the others by any of the moves a queen may make under the rules of the game.

Leonhard Euler studied the problem of the knight's tour, in which a knight must go to each square on the board just once (though the starting square may be visited twice in some statements of the problem). Another problem involving a solitary knight on an empty board is  to find the longest path so that the knight does not go over any previously trod squares.

\begin{thebibliography}{1}
\bibitem{am} A. Morehead \& G. Mott-Smith {\it Hoyle's Rules of Games} 3rd. Edition. New York: Penguin Group (2001): 309 - 317
\end{thebibliography}
%%%%%
%%%%%
\end{document}
