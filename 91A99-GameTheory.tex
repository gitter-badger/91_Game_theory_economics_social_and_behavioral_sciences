\documentclass[12pt]{article}
\usepackage{pmmeta}
\pmcanonicalname{GameTheory}
\pmcreated{2013-03-22 12:51:55}
\pmmodified{2013-03-22 12:51:55}
\pmowner{Henry}{455}
\pmmodifier{Henry}{455}
\pmtitle{game theory}
\pmrecord{6}{33202}
\pmprivacy{1}
\pmauthor{Henry}{455}
\pmtype{Topic}
\pmcomment{trigger rebuild}
\pmclassification{msc}{91A99}

\endmetadata

% this is the default PlanetMath preamble.  as your knowledge
% of TeX increases, you will probably want to edit this, but
% it should be fine as is for beginners.

% almost certainly you want these
\usepackage{amssymb}
\usepackage{amsmath}
\usepackage{amsfonts}

% used for TeXing text within eps files
%\usepackage{psfrag}
% need this for including graphics (\includegraphics)
%\usepackage{graphicx}
% for neatly defining theorems and propositions
%\usepackage{amsthm}
% making logically defined graphics
%%%\usepackage{xypic}

% there are many more packages, add them here as you need them

% define commands here
\begin{document}
Game theory is the study of games in a formalized setting.  Games are broken down into players and rules which define what the players can do and how much the players want each outcome.

Typically, game theory assumes that players are rational, a requirement which that players always make the decision which most benefits them based on the information available (as defined by that game), but also that players are always capable of making that decision (regardless of the amount of calculation which might be necessary in practice).

Branches of game theory include cooperative game theory, in which players can negotiate and enforce bargains and non-cooperative game theory, in which the only meaningful agreements are those which are ``self-enforcing,'' that is, which the players have an incentive not to break.

Many fields of mathematics (set theory, recursion theory, topology, and combinatorics, among others) apply game theory by representing problems as games and then use game theoretic techniques to find a solution.  (To see how an application might work, consider that a proof can be viewed as a game between a "prover" and a "refuter," where every universal quantifier represents a move by the refuter, and every existenial one a move by the prover; the proof is valid exactly when the prover can always win the corresponding game.)
%%%%%
%%%%%
\end{document}
