\documentclass[12pt]{article}
\usepackage{pmmeta}
\pmcanonicalname{Blackjack}
\pmcreated{2013-03-22 16:41:02}
\pmmodified{2013-03-22 16:41:02}
\pmowner{PrimeFan}{13766}
\pmmodifier{PrimeFan}{13766}
\pmtitle{blackjack}
\pmrecord{6}{38895}
\pmprivacy{1}
\pmauthor{PrimeFan}{13766}
\pmtype{Feature}
\pmcomment{trigger rebuild}
\pmclassification{msc}{91A05}
\pmsynonym{vingt-et-un}{Blackjack}

% this is the default PlanetMath preamble.  as your knowledge
% of TeX increases, you will probably want to edit this, but
% it should be fine as is for beginners.

% almost certainly you want these
\usepackage{amssymb}
\usepackage{amsmath}
\usepackage{amsfonts}

% used for TeXing text within eps files
%\usepackage{psfrag}
% need this for including graphics (\includegraphics)
%\usepackage{graphicx}
% for neatly defining theorems and propositions
%\usepackage{amsthm}
% making logically defined graphics
%%%\usepackage{xypic}

% there are many more packages, add them here as you need them

% define commands here

\begin{document}
{\em Blackjack} or {\em twenty-one} or {\em vingt-et-un} is a multi-player game of skill and luck played with standard playing cards and colored chips. The game may be played with two decks of cards instead of just one, especially if there are more than five players. The game begins with the dealer dealing each player a card face down and the players placing their initial bets. After that, players receive additional cards from the dealer, the goal being to collect a total value of exactly 21, no more.

The aces may be worth 1 or 11, as the player wishes, while the royal cards (jack, queen, king) are all worth 10 each.

There are 792 unrestricted integer partitions of 21, but we can disregard those containing integers greater than 11, as well as those containing more than four instances of an integer from 1 to 5 (assuming only one deck of cards is being used).

\begin{thebibliography}{1}
\bibitem{am} A. Morehead \& G. Mott-Smith {\it Play According to Hoyle: Hoyle's Rules of Games} New York: Signet (1963): 174 - 177
\end{thebibliography}
%%%%%
%%%%%
\end{document}
