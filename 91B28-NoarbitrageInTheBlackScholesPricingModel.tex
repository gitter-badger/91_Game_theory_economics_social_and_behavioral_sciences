\documentclass[12pt]{article}
\usepackage{pmmeta}
\pmcanonicalname{NoarbitrageInTheBlackScholesPricingModel}
\pmcreated{2013-03-22 16:54:04}
\pmmodified{2013-03-22 16:54:04}
\pmowner{stevecheng}{10074}
\pmmodifier{stevecheng}{10074}
\pmtitle{no-arbitrage in the Black-Scholes pricing model}
\pmrecord{4}{39158}
\pmprivacy{1}
\pmauthor{stevecheng}{10074}
\pmtype{Topic}
\pmcomment{trigger rebuild}
\pmclassification{msc}{91B28}
\pmclassification{msc}{60H10}

\endmetadata

% The standard font packages
\usepackage{amssymb}
\usepackage{amsmath}
\usepackage{amsfonts}

% For neatly defining theorems and definitions
%\usepackage{amsthm}

% Including EPS/PDF graphics (\includegraphics)
%\usepackage{graphicx}

% Making matrix-based graphics
%%%\usepackage{xypic}

% Enumeration lists with different styles
\usepackage{enumerate}

% Set up the theorem environments
%\newtheorem{thm}{Theorem}
%\newtheorem*{thm*}{Theorem}

\providecommand{\defnterm}[1]{\emph{#1}}

% The standard number systems
\newcommand{\complex}{\mathbb{C}}
\newcommand{\real}{\mathbb{R}}
\newcommand{\rat}{\mathbb{Q}}
\newcommand{\nat}{\mathbb{N}}
\newcommand{\intset}{\mathbb{Z}}

% Absolute values and norms
% Normal, wide, and big versions of the delimeters
\providecommand{\abs}[1]{\lvert#1\rvert}
\providecommand{\absW}[1]{\left\lvert#1\right\rvert}
\providecommand{\absB}[1]{\Bigl\lvert#1\Bigr\rvert}
\providecommand{\norm}[1]{\lVert#1\rVert}
\providecommand{\normW}[1]{\left\lVert#1\right\rVert}
\providecommand{\normB}[1]{\Bigl\lVert#1\Bigr\rVert}

% Differentiation operators
\providecommand{\od}[2]{\frac{d #1}{d #2}}
\providecommand{\pd}[2]{\frac{\partial #1}{\partial #2}}
\providecommand{\pdd}[2]{\frac{\partial^2 #1}{\partial #2}}
\providecommand{\ipd}[2]{\partial #1 / \partial #2}

% Differentials on integrals
\newcommand{\dx}{\, dx}
\newcommand{\dt}{\, dt}
\newcommand{\dmu}{\, d\mu}

% Inner products
\providecommand{\ip}[2]{\langle {#1}, {#2} \rangle}

% Calligraphic letters
\newcommand{\sF}{\mathcal{F}}
\newcommand{\sD}{\mathcal{D}}

% Standard spaces
\newcommand{\Hilb}{\mathcal{H}}
\newcommand{\Le}{\mathbf{L}}

% Operators and functions occassionally used in my articles
\DeclareMathOperator{\D}{D}
\DeclareMathOperator{\linspan}{span}
\DeclareMathOperator{\rank}{rank}
\DeclareMathOperator{\lindim}{dim}
\DeclareMathOperator{\sinc}{sinc}

% Probability stuff
\newcommand{\PP}{\mathbb{P}}
\newcommand{\E}{\mathbb{E}}

\newcommand{\EQ}{\mathbb{E}^\mathbb{Q}}
\newcommand{\PQ}{\mathbb{Q}}


\begin{document}
The underlying intuition behind the mechanics of the Black-Scholes formula,
in simpler cases, can be described without recourse to stochastic calculus.

The two basic principles
behind financial derivative pricing are:

\begin{enumerate}[i]
\item
A portfolio must be self-financing: ``you cannot create money out of thin air''
\item
There should be no opportunities for \emph{arbitrage} (a profit for any
one of the parties involving absolutely no risk).
\end{enumerate}

In the mathematically rigorous derivation, principle (i) is formalized
as the stochastic differential equation on the value of the portfolio $V(t)$
at time $t$:
\begin{align}\label{eq:port-self-financing}
d V(t) = \Delta(t) \, dX(t) + \Theta(t) \, dM(t)\,.
\end{align}

Principle (ii) was not particularly emphasized in the formal derivation,
but is a necessary consequence of our method of pricing a financial 
derivative instrument.  Consider a contract agreed between two 
parties, at time $0$, that stipulates the seller to make a payment of $V(T)$
to the buyer of the contract at time $T$; 
the buyer pays $V(0)$ at time $0$ to take advantage of this offer.
The value $V(T)$, in general, depends on some risky (or \emph{volatile})
asset such as a stock, so we cannot easily ascertain that the deal
is fair or not.

But there is still a concept of a fair value for the contract.
If the seller, having sold the contract, starts to immediately
manage his investments in such a way as to mitigate \emph{all} of the risk,
then the initial capital for managing such investment must be the
fair value of the contract.  If the value paid by the buyer to the seller
at time $0$ is larger than this fair value, then the seller makes a 
\emph{risk-free} profit: he uses up some amount of the up-front fee to manage
his investment portfolio, and he will always be able to meet his obligation
to pay the buyer $V(T)$ at time $T$, yet still have some amount left over
from the original fee that he can pocket.

Conversely, if the fair value turns out to be lower than the up-front fee offered,
the buyer has taken advantage of the seller.

In the formal derivation of the Black-Scholes formula,
the portfolio value $V(t)$ at varying times $0 \leq t \leq T$
represents the investment management decision,
subject to the self-financing condition represented by
equation \eqref{eq:port-self-financing}.
The terminal condition 
of the portfolio value $V(T)$ at time $T$ is given and fixed
to be exactly the seller's obligation at that time; hence the seller
of the contract is in a riskless position if he follows the appropriate
investment decisions.
Moreover, the stochastic differential equation 
has a unique solution $V(0)$ at time $0$: this is the initial capital required
for the investment strategy that hedges away all risk.
Thus, the method of solution of the stochastic differential equations
of the Black-Scholes model incorporate the \emph{no-arbitrage} principle.

\medskip

In simple contracts, we can find out, by elementary reasoning,
what the fair value should be.  

Consider a \emph{forward contract},
which is an agreement, made at time $0$, for the buyer to obtain an asset
at time $T$ in exchange for a payment of $F$ at time $T$.
Denote the value of the asset at various times by $X(t)$.  The value
of $X(0)$ is known at the time of signing of the contract, 
but $X(T)$ is a random variable, because the asset price
is volatile.

The fair value of the forward contract
is the value that would give the buyer and seller no distinct
advantage compared to the situation when the buyer
simply buys and pays for the asset at time $0$.\footnote{So why does not the buyer simply acquire the asset at time $0$?  Because in practice, it may be inconvenient 
or costly for the buyer to hold on to an asset that is not going to be used
until at time $T$.  This is not much of an issue for stocks, which are paper 
assets, but it matters if the asset is a commodity such as energy --- 
the buyer might not be able to store these, or need a significant
amount of money to do so.  Of course, in the theoretical reason presented here,
we ignore all such transactional costs.}
If we are buying the asset, we can either get that asset now (at time $0$) 
for a price of $X(0)$, or decide to get it later (at time $T$),
meanwhile stashing the equivalent amount of money ($X(0)$)
in the bank.  In the latter situation, when we do get the asset at time $T$,
the money in our bank account is now $X(0) e^{rT}$, assuming
continuous compounding at an interest rate of $r$.   

Hence we are prepared to pay $X(0)e^{rT}$ when buying the asset \emph{forward}.
This is the price of the forward contract.
Note that it does not involve the price $X(T)$, because that price
is random and unknown when we sign the contract.  It is not even
the expected future price $\E{X(T)}$, because there is certainly
no guarantee that $X(T)$ will take on the value $\E{X(T)}$.

(more to be written...)

%%%%%
%%%%%
\end{document}
