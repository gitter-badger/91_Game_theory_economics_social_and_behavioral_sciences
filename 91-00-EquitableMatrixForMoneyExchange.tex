\documentclass[12pt]{article}
\usepackage{pmmeta}
\pmcanonicalname{EquitableMatrixForMoneyExchange}
\pmcreated{2013-03-22 14:59:25}
\pmmodified{2013-03-22 14:59:25}
\pmowner{mathforever}{4370}
\pmmodifier{mathforever}{4370}
\pmtitle{equitable matrix for money exchange}
\pmrecord{9}{36695}
\pmprivacy{1}
\pmauthor{mathforever}{4370}
\pmtype{Example}
\pmcomment{trigger rebuild}
\pmclassification{msc}{91-00}
\pmclassification{msc}{91B28}
\pmclassification{msc}{15-00}

% this is the default PlanetMath preamble.  as your knowledge
% of TeX increases, you will probably want to edit this, but
% it should be fine as is for beginners.

% almost certainly you want these
\usepackage{amssymb}
\usepackage{amsmath}
\usepackage{amsfonts}
\usepackage{amsthm}

\usepackage{mathrsfs}

% used for TeXing text within eps files
%\usepackage{psfrag}
% need this for including graphics (\includegraphics)
%\usepackage{graphicx}
% for neatly defining theorems and propositions
%
% making logically defined graphics
%%%\usepackage{xypic}

% there are many more packages, add them here as you need them

% define commands here

\theoremstyle{definition}
\newtheorem{exm}{Example}
\newcommand{\mc}{\centering}
\begin{document}
This example shows how equitable matrices arise in money exchange.
Consider $n$ currencies $C_1,C_2,\ldots,C_n$, where $C_i$ stays for some
name like ``dollar'', ``euro'', ``pound'', etc. Denote the \textit{exchange rate}
between currencies $C_i$ and $C_j$ as $a_{ij}>0$, i.e.
\begin{equation}
1\ C_i \longrightarrow a_{ij}\ C_j.
\label{er}
\end{equation}
We will call $A=(a_{ij})_{i,j=1}^n$ an \textit{exchange rates matrix}.
Suppose, $A$ is an
equitable matrix, i.e.
%
\begin{equation}
    a_{ij} = a_{ik} \cdot a_{kj}, \quad i,j,k=1,\ldots, n.
\label{eqp}
\end{equation}
%
Let us discuss consequences of this. First of all, there is no loss when exchanging
between two currencies: if one exchanges $u$ units of $C_i$ to $C_j$
and then back to $C_i$, one will have again $u$ units. Indeed,
$$
    u\ C_i \longrightarrow u\cdot a_{ij}\ C_j \longrightarrow u\cdot(a_{ij}\cdot a_{ji})\ C_i
$$
and desired conjecture follows from the fact that
%
\begin{equation}
    a_{ij} \cdot a_{ji}=1, \quad i,j=1,\ldots, n,
\label{lesseqp}
\end{equation}
%
which can be proven by putting $j=i$ in (\ref{eqp}) and using diagonal property
($a_{ii}=1$, for all $i=1,\ldots,n$).
But equitable property \eqref{eqp} suggests in fact more than just \eqref{lesseqp}, i.e. more
than just no loss by changing from one currency to another and back. For illustration consider
an example.

\begin{exm}
    Let us take three currencies $C_1$, $C_2$, $C_3$ with the following exchange rates
    \begin{eqnarray*}
        1\ C_1 & \longrightarrow & 2\ C_2,\\
        1\ C_1 & \longrightarrow & 3\ C_3,\\
        1\ C_2 & \longrightarrow & 2\ C_3.
    \end{eqnarray*}
    The above relations define $a_{12},\ a_{13},\ a_{23}$ in the exchange rates matrix $A$.
    Let us define other elements by \eqref{lesseqp}. This gives us
    $$
    A=\left(
        \begin{array}{ccc}
            1 & 2 & 3\\
            1/2 & 1 & 2\\
            1/3 & 1/2 & 1
        \end{array}
        \right).
    $$
    Now assume one has 100 $C_3$ units and wants to exchange them to $C_1$. If one does this
    directly, one will obtain 300 $C_1$ units. But if one exchanges first to $C_2$ and then
    to $C_1$, one will obtain 400 $C_1$ units.
\end{exm}

For an equitable exchange rates matrix the above described situation is impossible:
exchanging $u\ C_i$ units to $C_j$ is the same as first exchanging to $C_k$ and then
to $C_j$. Indeed,
$$
\begin{array}{ccccc}
    & & u\ C_i & \longrightarrow & u\cdot a_{ij}\ C_j \\
    u\ C_i & \longrightarrow & u\cdot a_{ik}\ C_k & \longrightarrow & u\cdot(a_{ik}\cdot a_{kj})\ C_j
\end{array}
$$
and the final result is the same due to the equitable property \eqref{eqp}. Note, that the matrix
from the example is not equitable
$$
    a_{23}\neq a_{21}\cdot a_{13}.
$$

The above consideration shows that equitable property does not allow making money
just by exchanging currencies. If \eqref{eqp} does not hold for some
indexes, for example
$$
    a_{ij} < a_{ik}\cdot a_{kj},
$$
then having $u\ C_i$ units one can make money just by exchanging them to $C_j$
through $C_k$ and back
$$
    u\ C_i \longrightarrow u\cdot a_{ik}\ C_k \longrightarrow u\cdot(a_{ik}\cdot a_{kj})\ C_j
           \longrightarrow u\cdot\left( \frac{a_{ik}\cdot a_{kj}}{a_{ij}} \right)\ C_i.
$$
If we denote $q:=\frac{a_{ik}\cdot a_{kj}}{a_{ij}}>1$, then after making $N$ such
exchanges instead of $u$ one would have $u\cdot q^N$ units --- the capital would
increase like geometric progression! If there is an opposite inequality
$$
    a_{ij} > a_{ik}\cdot a_{kj},
$$
then such advantage have those with $C_j$ currency. So, condition \eqref{eqp} guarantees
that no one can speculate on currency exchange, thus motivating the name
``equitable'' --- ``to be fair''.

Let us give an interpretation of the matrix-vector multiplication for exchange rates matrices.
This interpretation is not connected with the equitable property \eqref{eqp} and shows
why exchange rates matrices are useful in general.

Consider a company which operates on the international market (e.g., a company which makes
furniture/cars/household equipment/etc, and sells their products to more than one country) and, thus, obtains
money in different currencies $C_1,C_2,...,C_n$. From time to time,
for such a company the natural question arises: what is the
total amount of money we have? Specifically, at a given time the company obtained
the following money: $u_1\ C_1,\; u_2\ C_2,...,\; u_n\ C_n$. With given exchange rates
$a_{ij}$, the total amount of money in the currency $C_i$ is
$\sum_{j=1}^n  u_j\cdot a_{ji}$.
That's why we have
\begin{quote}
    \textit{if matrix $A$ is the exchange rates matrix for currencies $C_1,...,C_n$,
    $u=(u_1,...,u_n)$ is a row-vector with components expressing amount of units in each currency,
    then components of the vector $u\cdot A$ express the total amount of money in each currency.}
\end{quote}
The following example gives illustration to this.

\begin{exm}
    Imagine a company located in Germany, let's call it ``Peaut'', which makes auto ``Leoptera''.
    It sells this auto to five different countries: Germany (its home country), USA, United Kingdom,
    Japan, and Switzerland. Thus, it needs to operate with the following currencies:
    \begin{eqnarray*}
        C_1 & = &\mbox{``euro''=``EUR''},\\
        C_2 & = &\mbox{``USA dollar''=``USD''},\\
        C_3 & = &\mbox{``British pound''=``GBP''},\\
        C_4 & = &\mbox{``Yapanese yen''=``JPY''},\\
        C_5 & = &\mbox{``Swiss frank''=``CHF''}.
    \end{eqnarray*}
    The corresponding exchange rates matrix $A$ is presented in Table~\ref{erm}.
    The values were first collected from
    \htmladdnormallink{Wikipedia}{http://en.wikipedia.org/}
    at the middle of 2005, and then they were modified such that the resulting matrix is equitable.
%
\begin{table}[ht]
%
\mc
\begin{tabular}{c|c|c|c|c|c|}
& \parbox[c]{1.5cm}{\mc EUR} & \parbox[c]{1.5cm}{\mc USD} & \parbox[c]{1.5cm}{\mc GBP} &
  \parbox[c]{1.5cm}{\mc JPY} & \parbox[c]{1.5cm}{\mc CHF} \\ \hline
\parbox[c]{1cm}{\mc EUR} &
\parbox[c]{1.5cm}{\mc 1} & \parbox[c]{1.5cm}{\mc 1.25} & \parbox[c]{1.5cm}{\mc 0.625} &
\parbox[c]{1.5cm}{\mc 125} & \parbox[c]{1.5cm}{\mc 1.5626}
\\ \hline
\parbox[c]{1cm}{\mc USD} &
\parbox[c]{1.5cm}{\mc 0.8} & \parbox[c]{1.5cm}{\mc 1} & \parbox[c]{1.5cm}{\mc 0.5} &
\parbox[c]{1.5cm}{\mc 100} & \parbox[c]{1.5cm}{\mc 1.25}
\\ \hline
\parbox[c]{1cm}{\mc GBP} &
\parbox[c]{1.5cm}{\mc 1.6} & \parbox[c]{1.5cm}{\mc 2} & \parbox[c]{1.5cm}{\mc 1} &
\parbox[c]{1.5cm}{\mc 200} & \parbox[c]{1.5cm}{\mc 2.5}
\\ \hline
\parbox[c]{1cm}{\mc JPY} &
\parbox[c]{1.5cm}{\mc 0.008} & \parbox[c]{1.5cm}{\mc 0.01} & \parbox[c]{1.5cm}{\mc 0.005} &
\parbox[c]{1.5cm}{\mc 1} & \parbox[c]{1.5cm}{\mc 0.0125}
\\ \hline
\parbox[c]{1cm}{\mc CHF} &
\parbox[c]{1.5cm}{\mc 0.64} & \parbox[c]{1.5cm}{\mc 0.8} & \parbox[c]{1.5cm}{\mc 0.4} &
\parbox[c]{1.5cm}{\mc 80} & \parbox[c]{1.5cm}{\mc 1}
\\ \hline
\end{tabular}
%
\caption{Exchange rate matrix used in the example.
\label{erm}}
\end{table}
%

    The price of ``Leoptera'' in each country, number of sold cars during a year, and
    corresponding amount of money are gathered in Table~\ref{u}. The last column gives
    components of the row-vector $u$ in our example. To answer the question what is the total
    amount of money are obtained by ``Peaut'', one needs to compute
    $u\cdot A$. The result is presented in Table~\ref{Au}.
%
\begin{table}[ht]
\mc
\begin{tabular}{c|c|c|c|}
\parbox[c]{2cm}{\mc Country where the cars were sold\vspace{0.1cm}} &
\parbox[c]{2cm}{\mc Price for one car} &
\parbox[c]{2.5cm}{\mc Amount of\\ sold cars\\ (in thousands)} &
\parbox[c]{2.5cm}{\mc Obtained money\\ (in milliards)}
\\ \hline
\parbox[c]{2cm}{\mc Germany} &
\parbox[c]{2cm}{\mc 21.000 EUR} &
\parbox[c]{2cm}{\mc 100} &
\parbox[c]{2cm}{\mc 2.1 EUR}
\\ \hline
\parbox[c]{2cm}{\mc USA} &
\parbox[c]{2cm}{\mc 32.000 USD} &
\parbox[c]{2cm}{\mc 35} &
\parbox[c]{2cm}{\mc 1.12 USD}
\\ \hline
\parbox[c]{2cm}{\mc UK} &
\parbox[c]{2cm}{\mc 14.000 GBP} &
\parbox[c]{2cm}{\mc 40} &
\parbox[c]{2cm}{\mc 0.56 GBP}
\\ \hline
\parbox[c]{2cm}{\mc Japan} &
\parbox[c]{2cm}{\mc 3.1 mln JPY} &
\parbox[c]{2cm}{\mc 8} &
\parbox[c]{2cm}{\mc 24.8 JPY}
\\ \hline
\parbox[c]{2cm}{\mc Switzerland} &
\parbox[c]{2cm}{\mc 34.000 CHF} &
\parbox[c]{2cm}{\mc 5} &
\parbox[c]{2cm}{\mc 0.17 CHF}
\\ \hline
\end{tabular}
%
\caption{Statistic of selling ``Leoptera'' during a year.
\label{u}}
\end{table}
%
\begin{table}[ht]
\mc
\begin{tabular}{c|c|}
\parbox[c]{2cm}{\mc Country's currency} &
\parbox[c]{3cm}{\mc The total amount of money in different currencies\\ (in milliards)}
\\ \hline
\parbox[c]{2cm}{\mc EUR} &
\parbox[c]{2cm}{\mc 4.1992}
\\ \hline
\parbox[c]{2cm}{\mc USD} &
\parbox[c]{2cm}{\mc 5.249}
\\ \hline
\parbox[c]{2cm}{\mc GBP} &
\parbox[c]{2cm}{\mc 2.6245}
\\ \hline
\parbox[c]{2cm}{\mc JPY} &
\parbox[c]{2cm}{\mc 524.9}
\\ \hline
\parbox[c]{2cm}{\mc CHF} &
\parbox[c]{2cm}{\mc 6.56146}
\\ \hline
\end{tabular}
%
\caption{Total amount of money obtained by ``Peaut''.
\label{Au}}
\end{table}
%

\end{exm}
%%%%%
%%%%%
\end{document}
