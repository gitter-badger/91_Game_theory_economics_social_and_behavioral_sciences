\documentclass[12pt]{article}
\usepackage{pmmeta}
\pmcanonicalname{RangeVoting}
\pmcreated{2013-03-22 16:10:00}
\pmmodified{2013-03-22 16:10:00}
\pmowner{Mathprof}{13753}
\pmmodifier{Mathprof}{13753}
\pmtitle{range voting}
\pmrecord{51}{38250}
\pmprivacy{1}
\pmauthor{Mathprof}{13753}
\pmtype{Topic}
\pmcomment{trigger rebuild}
\pmclassification{msc}{91F10}
\pmdefines{approval voting}
\pmdefines{single-digit range voting}
\pmdefines{two-digit range voting}

% this is the default PlanetMath preamble.  as your knowledge
% of TeX increases, you will probably want to edit this, but
% it should be fine as is for beginners.

% almost certainly you want these
\usepackage{amssymb}
\usepackage{amsmath}
\usepackage{amsfonts}

% used for TeXing text within eps files
%\usepackage{psfrag}
% need this for including graphics (\includegraphics)
%\usepackage{graphicx}
% for neatly defining theorems and propositions
%\usepackage{amsthm}
% making logically defined graphics
%%%\usepackage{xypic}

% there are many more packages, add them here as you need them

% define commands here

\begin{document}
\title{Range voting}
\author{Warren D. Smith}

%\begin{abstract}
%A short introduction to range voting.
%\end{abstract}

%\begin{Keywords}
%\end{Keywords}
\PMlinkescapeword{theorem}
\PMlinkescapeword{models}
\PMlinkescapeword{definition}
\PMlinkescapeword{range}
\PMlinkescapeword{site}
\PMlinkescapeword{reduced}
\PMlinkescapeword{strategy}
\PMlinkescapeword{point}
\PMlinkescapeword{transform}
\PMlinkescapeword{score}\PMlinkescapeword{scores}
\PMlinkescapeword{scenario}\PMlinkescapeword{difference}
\PMlinkescapeword{model}\PMlinkescapeword{sites}
\PMlinkescapeword{group}
\PMlinkescapeword{browser}
\PMlinkescapeword{average}
\PMlinkescapeword{order}
\PMlinkescapeword{term}
\PMlinkescapeword{paradox}\PMlinkescapeword{paradoxes}
\PMlinkescapeword{join}
\PMlinkescapeword{associate}
\PMlinkescapeword{measure}\PMlinkescapeword{measures}
\PMlinkescapeword{variety}
\PMlinkescapeword{browsing}
\PMlinkescapeword{comparable}
\PMlinkescapeword{paradox}
\PMlinkescapeword{sound}
\PMlinkescapeword{equivalent}
\PMlinkescapeword{behavior}
\PMlinkescapeword{information}

\emph{Range voting}
is a single-winner voting system in which each vote consists of one numerical score 
from 0 to U awarded to each candidate,
where U is some positive number. 
In \emph{single-digit range voting} U is 9, 
while \emph{two-digit range voting} has U=99.
(One could also consider permitting the interval to be, e.g. $[-10, 10]$.)
The candidate with the highest average score wins. 
In a common variant, ``X'' scores also are permitted for candidates 
(meaning:``I wish to express no opinion about this candidate''). 
For example  (46,0,32,99)  could be one vote in a 4-candidate two-digit range voting
election; advise 99/0 for best/worst.

{\bf Approval voting} \cite{BramsFAV} is the degenerate form of range voting in which
there are only two allowed scores:  0 and 1.


\begin{thebibliography}{9}



\bibitem{BramsFAV}
S.J. Brams \& P. Fishburn:  Approval voting, Birkhauser 1983.


\end{thebibliography}

\end{document}

%%%%%
%%%%%
\end{document}
