\documentclass[12pt]{article}
\usepackage{pmmeta}
\pmcanonicalname{ExampleOfNashEquilibrium}
\pmcreated{2013-03-22 12:52:48}
\pmmodified{2013-03-22 12:52:48}
\pmowner{Henry}{455}
\pmmodifier{Henry}{455}
\pmtitle{example of Nash equilibrium}
\pmrecord{6}{33222}
\pmprivacy{1}
\pmauthor{Henry}{455}
\pmtype{Example}
\pmcomment{trigger rebuild}
\pmclassification{msc}{91A99}

% this is the default PlanetMath preamble.  as your knowledge
% of TeX increases, you will probably want to edit this, but
% it should be fine as is for beginners.

% almost certainly you want these
\usepackage{amssymb}
\usepackage{amsmath}
\usepackage{amsfonts}

% used for TeXing text within eps files
%\usepackage{psfrag}
% need this for including graphics (\includegraphics)
%\usepackage{graphicx}
% for neatly defining theorems and propositions
%\usepackage{amsthm}
% making logically defined graphics
%%%\usepackage{xypic}

% there are many more packages, add them here as you need them

% define commands here
%\PMlinkescapeword{theory}
\begin{document}
Consider the first two games given as examples of normal form games.

In Prisoner's Dilemma the only Nash equilibrium is for both players to play $D$: it's apparent that, no matter what player $1$ plays, player $2$ does better playing $D$, and vice-versa for $1$.

Battle of the Sexes has three Nash equilibria.  Both $(O,O)$ and $(F,F)$ are Nash equilibria, since it should be clear that if player $2$ expects player $1$ to play $O$, player $2$ does best by playing $O$, and vice-versa, while the same situation holds if player $2$ expects player $1$ to play $F$.  The third is a mixed equilibrium; player $1$ plays $O$ with $\frac{2}{3}$ probability and player $2$ plays $O$ with $\frac{1}{3}$ probability.  We confirm that these are equilibria by testing the first derivatives (if $0$ then the strategy is either maximal or minimal).  Technically we also need to check the second derivative to make sure that it is a maximum, but with simple games this is not really necessary.

Let player $1$ play $O$ with probability $p$ and player $2$ plays $O$ with probability $q$.

\begin{displaymath}
u_1(p,q)=2pq+(1-p)(1-q)=2pq-p-q+pq=3pq-p-q
\end{displaymath}
\begin{displaymath}
u_2(p,q)=pq+2(1-p)(1-q)=3pq-2p-2q
\end{displaymath}

\begin{displaymath}
\frac{\partial u_1(p,q)}{\partial p}=3q-1
\end{displaymath}
\begin{displaymath}
\frac{\partial u_2(p,q)}{\partial q}=3p-2
\end{displaymath}

And indeed the derivatives are $0$ at $p=\frac{2}{3}$ and $q=\frac{1}{3}$.
%%%%%
%%%%%
\end{document}
