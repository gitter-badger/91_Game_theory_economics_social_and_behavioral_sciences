\documentclass[12pt]{article}
\usepackage{pmmeta}
\pmcanonicalname{Voting}
\pmcreated{2013-03-22 16:49:14}
\pmmodified{2013-03-22 16:49:14}
\pmowner{PrimeFan}{13766}
\pmmodifier{PrimeFan}{13766}
\pmtitle{voting}
\pmrecord{4}{39058}
\pmprivacy{1}
\pmauthor{PrimeFan}{13766}
\pmtype{Topic}
\pmcomment{trigger rebuild}
\pmclassification{msc}{91A99}

% this is the default PlanetMath preamble.  as your knowledge
% of TeX increases, you will probably want to edit this, but
% it should be fine as is for beginners.

% almost certainly you want these
\usepackage{amssymb}
\usepackage{amsmath}
\usepackage{amsfonts}

% used for TeXing text within eps files
%\usepackage{psfrag}
% need this for including graphics (\includegraphics)
%\usepackage{graphicx}
% for neatly defining theorems and propositions
%\usepackage{amsthm}
% making logically defined graphics
%%%\usepackage{xypic}

% there are many more packages, add them here as you need them

% define commands here

\begin{document}
{\em Voting} is the process in which a group of persons or entities in an organization are polled in order to make a binding decision. The decision is usually to appoint someone from among a pool of candidates to an office in the organization, or to change a rule of the organization. For most elected government offices, each citizen of voting age gets one vote. Elections of officers of professional organizations usually use a system of weighted choices, while elections of officers in a publically-owned corporation usually give each shareholder as many votes as he has shares.

In some cases, to win it is enough for a candidate or proposal to receive more votes than any other candidate or proposal. For example, four candidates run for Mayor of Anytown USA, which has about 1000 registered voters. Anderson gets 490 votes, Brown, Clark and Davies each get about a third of the remaining votes. In such a scenario, Anderson would win unless the rules of the election require a simple majority (more than half the votes cast). In some special cases, a $\frac{2}{3}$ majority is required, such as ratification of amendments to the United States Constitution.

For the office of President of the United States, it is not actually the people who vote but the States. The people of each state actually vote to decide how the state will vote in the Electoral College. To give an actual example, in the 2000 election, Al Gore got about half a million more votes than George W. Bush at the level of the people in each state counted in one total. However, after the Supreme Court decision regarding Florida, Bush received 271 of the Electoral College votes and Gore just 266; thus Bush won with the smallest possible majority under the rules of presidential election.

Even in the absence of tampering, or suspicion of tampering, there can be many paradoxes facing voters, whether they can cast votes for a single candidate or for multiple candidates in order of preference. There are cases in which a voter would rather vote for whom ``they would prefer to win the election,'' and cases in which ``it may be rational not to do so,'' for example, when a voter has not just ``a preference for one party, but a strong distaste for another.'' (Robertson, 2004)Hoffman calls it ``insincere voting'' when a voter strategically votes contrary to their own preferences in order to secure their prefered result.

Hoffman gives an example in which the American Mathematical Society voted for members to send on a special committee, using a system of voting devised by Thomas Hare in which voters vote for candidates in order of preference. With his system, Hare intended to correct an anomaly in England where some districts electing more than one candidate could completely disenfranchise some minority voters. Thus, in the Hare system, a candidate must obtain a quota of the first-choice votes, with ``the quota computed to be the minimum number of first place votes such that the maximum number of candidates who could meet the quota corresponds to the number of open seats.'' (Hoffman, 1988) The instructions for the AMS ballot said that ``no tactical advantage is to be gained by marking fewer candidates.'' Game theorist Steven Brams constructed an example in which precisely such an advantage can be gained. Then there's something called the ``Alabama paradox,'' in which ``a state can lose representatives in a larger [House of Representatives].'' Hoffman concludes that a voting ``method that is almost always free of paradoxes is clearly preferable to one riddled with them.''

\begin{thebibliography}{3}
\bibitem{mg} Martin Gardner, {\it The Sixth Book of Mathematical Games from Scientific American}. Chicago: University of Chicago Press (1984): 25
\bibitem{ph} Paul Hoffman, {\it Archimedes' Revenge: The Joys and Perils of Mathematics} New York: Fawcett Crest (1988): 213 - 260
\bibitem{dr} David Robertson, ``Tactical Voting'' in {\it The Routledge Dictionary of Politics} London: Taylor \& Francis Group (2004)
\end{thebibliography}

\subsection{External link}

\PMlinkexternal{Electronic Voting Demonstration}{http://bobspoetry.com/eVoteDemo.html} A demonstration of how voting machines can be rigged to favor a particular candidate. However, the Javascript source code of the demonstration is easily exposed by a Web browser, while the source code for the firmware of the Diebold voting machines is a trade secret.
%%%%%
%%%%%
\end{document}
