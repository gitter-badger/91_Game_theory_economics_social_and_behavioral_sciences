\documentclass[12pt]{article}
\usepackage{pmmeta}
\pmcanonicalname{Game}
\pmcreated{2013-03-22 12:51:19}
\pmmodified{2013-03-22 12:51:19}
\pmowner{Henry}{455}
\pmmodifier{Henry}{455}
\pmtitle{game}
\pmrecord{12}{33189}
\pmprivacy{1}
\pmauthor{Henry}{455}
\pmtype{Definition}
\pmcomment{trigger rebuild}
\pmclassification{msc}{91A99}
\pmrelated{CommonKnowledge}
\pmrelated{NormalFormGame}
\pmrelated{CompleteInformation}
\pmrelated{Strategy2}
\pmdefines{outcome}
\pmdefines{game}
\pmdefines{player}
\pmdefines{utility function}
\pmdefines{payoff}

% this is the default PlanetMath preamble.  as your knowledge
% of TeX increases, you will probably want to edit this, but
% it should be fine as is for beginners.

% almost certainly you want these
\usepackage{amssymb}
\usepackage{amsmath}
\usepackage{amsfonts}

% used for TeXing text within eps files
%\usepackage{psfrag}
% need this for including graphics (\includegraphics)
%\usepackage{graphicx}
% for neatly defining theorems and propositions
%\usepackage{amsthm}
% making logically defined graphics
%%%\usepackage{xypic}

% there are many more packages, add them here as you need them

% define commands here
\begin{document}
In general, a \emph{game} is a way of describing a situation in which \emph{players} make choices with the intent of optimizing their \emph{utility}.  Formally, a game includes three features:
\begin{itemize}

\item a set of pure strategies for each player (their strategy space)

\item a way of determining an \emph{outcome} from the strategies selected by the players

\item a utility function for each player specifying their \emph{payoff} for each outcome

\end{itemize}
%%%%%
%%%%%
\end{document}
