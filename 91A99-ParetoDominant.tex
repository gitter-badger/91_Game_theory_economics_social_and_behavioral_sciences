\documentclass[12pt]{article}
\usepackage{pmmeta}
\pmcanonicalname{ParetoDominant}
\pmcreated{2013-03-22 12:51:32}
\pmmodified{2013-03-22 12:51:32}
\pmowner{Henry}{455}
\pmmodifier{Henry}{455}
\pmtitle{Pareto dominant}
\pmrecord{6}{33194}
\pmprivacy{1}
\pmauthor{Henry}{455}
\pmtype{Definition}
\pmcomment{trigger rebuild}
\pmclassification{msc}{91A99}
\pmdefines{strongly Pareto optimal}
\pmdefines{weakly Pareto optimal}
\pmdefines{strongly Pareto dominates}
\pmdefines{strongly Pareto dominant}
\pmdefines{Pareto dominates}
\pmdefines{Pareto dominant}
\pmdefines{weakly Pareto dominates}
\pmdefines{weakly Pareto dominant}

\endmetadata

% this is the default PlanetMath preamble.  as your knowledge
% of TeX increases, you will probably want to edit this, but
% it should be fine as is for beginners.

% almost certainly you want these
\usepackage{amssymb}
\usepackage{amsmath}
\usepackage{amsfonts}

% used for TeXing text within eps files
%\usepackage{psfrag}
% need this for including graphics (\includegraphics)
%\usepackage{graphicx}
% for neatly defining theorems and propositions
%\usepackage{amsthm}
% making logically defined graphics
%%%\usepackage{xypic}

% there are many more packages, add them here as you need them

% define commands here
\begin{document}
An outcome $s^*$ \emph{strongly Pareto dominates} $s^\prime$ if:
\begin{displaymath}
\forall i\leq n \left[u_i(s^*)>u_i(s^\prime)\right]
\end{displaymath}

An outcome $s^*$ \emph{weakly Pareto dominates} $s^\prime$ if:
\begin{displaymath}
\forall i\leq n \left[u_i(s^*)\geq u_i(s^\prime)\right]
\end{displaymath}

$s^*$ is \emph{strongly Pareto optimal} if whenever $s^\prime$ weakly Pareto dominates $s^*$, $\forall i\leq n\left[u_i(s^*)=u_i(s^\prime)\right]$.  That is, there is no strategy which provides at least as large a payoff to each player and a larger one to at least one.
$s^*$ is \emph{weakly Pareto optimal} if there is no $s^\prime$ such that $s^\prime$ strongly Pareto dominates $s^*$.
%%%%%
%%%%%
\end{document}
