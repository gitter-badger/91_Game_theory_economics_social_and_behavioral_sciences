\documentclass[12pt]{article}
\usepackage{pmmeta}
\pmcanonicalname{NashEquilibrium}
\pmcreated{2013-03-22 12:51:35}
\pmmodified{2013-03-22 12:51:35}
\pmowner{Henry}{455}
\pmmodifier{Henry}{455}
\pmtitle{Nash equilibrium}
\pmrecord{8}{33195}
\pmprivacy{1}
\pmauthor{Henry}{455}
\pmtype{Definition}
\pmcomment{trigger rebuild}
\pmclassification{msc}{91A99}

\endmetadata

% this is the default PlanetMath preamble.  as your knowledge
% of TeX increases, you will probably want to edit this, but
% it should be fine as is for beginners.

% almost certainly you want these
\usepackage{amssymb}
\usepackage{amsmath}
\usepackage{amsfonts}

% used for TeXing text within eps files
%\usepackage{psfrag}
% need this for including graphics (\includegraphics)
%\usepackage{graphicx}
% for neatly defining theorems and propositions
%\usepackage{amsthm}
% making logically defined graphics
%%%\usepackage{xypic}

% there are many more packages, add them here as you need them

% define commands here
\begin{document}
A Nash equilibrium of a game is a set of (possibly mixed) strategies $\sigma=(\sigma_1,\ldots,\sigma_n)$ such that, if each player $i$ believes that that every other player $j$ will play $\sigma_j$, then $i$ should play $\sigma_i$.  That is, when $u_i$ is the utility function for the $i$-th player:
\begin{displaymath}
\sigma_i\neq \sigma^\prime_i\rightarrow u_i(\sigma_i,\sigma_{-1})>u_i(\sigma^\prime_i,\sigma_{-1})
\end{displaymath}
\begin{displaymath}
\forall i\leq n \text{ and } \forall \sigma^\prime_i\in \Sigma_i
\end{displaymath}

Translated, this says that if any player plays any strategy other than the one in the Nash equilibrium then that player would do worse than playing the Nash equilibrium.
%%%%%
%%%%%
\end{document}
